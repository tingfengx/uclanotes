\documentclass[11pt]{book}

\usepackage{geometry}
\usepackage{amssymb}
\usepackage[
    xetex, 
    dvipsnames
]{xcolor}
\usepackage[
    colorlinks=true,
    linkcolor=black,
    urlcolor=Thistle
]{hyperref}
\usepackage{fancyhdr}
\usepackage{datetime}
\usepackage{xargs}
\usepackage{ccicons}
\usepackage{mdframed}
\usepackage{caption}
\usepackage{cancel}
\usepackage{parskip}
\usepackage[nottoc]{tocbibind}
%\usepackage[
%    outputdir=.texpadtmp
%]{minted}

% ==== License =====
\usepackage[
    type={CC}, 
    modifier={by-nc-sa}, 
    version={4.0},
]{doclicense}

% ==== set font ====
\usepackage{amsmath}
\usepackage{unicode-math}
%\setmainfont{texgyrepagella-regular.otf}
\setmainfont{Palatino}
\setmathfont{texgyrepagella-math.otf}

% ==== todo notes ====
\usepackage[
    colorinlistoftodos,
    prependcaption,
    textsize=tiny
]{todonotes}
\newcommandx{\note}[2][1=]{\todo[linecolor=Thistle,backgroundcolor=Thistle!25,bordercolor=Thistle,#1]{#2}}
\newcommandx{\unsure}[2][1=]{\todo[linecolor=red,backgroundcolor=red!25,bordercolor=red,#1]{#2}}
\newcommandx{\change}[2][1=]{\todo[linecolor=blue,backgroundcolor=blue!25,bordercolor=blue,#1]{#2}}
\newcommandx{\info}[2][1=]{\todo[linecolor=OliveGreen,backgroundcolor=OliveGreen!25,bordercolor=OliveGreen,#1]{#2}}

% General
\newcommand{\mc}[1]{\mathcal{#1}}

% Math Bold Font, Vector Notations
\newcommand{\ba}{\mathbf{a}}
\newcommand{\bb}{\mathbf{b}}
\newcommand{\bc}{\mathbf{c}}
\newcommand{\bd}{\mathbf{d}}
\newcommand{\be}{\mathbf{e}}
\renewcommand{\bf}{\mathbf{f}}
\newcommand{\bg}{\mathbf{g}}
\newcommand{\bh}{\mathbf{h}}
\newcommand{\bi}{\mathbf{i}}
\newcommand{\bj}{\mathbf{j}}
\newcommand{\bk}{\mathbf{k}}
\newcommand{\bl}{\mathbf{l}}
\newcommand{\bm}{\mathbf{m}}
\newcommand{\bn}{\mathbf{n}}
\newcommand{\bo}{\mathbf{o}}
\newcommand{\bp}{\mathbf{p}}
\newcommand{\bq}{\mathbf{q}}
\newcommand{\br}{\mathbf{r}}
\newcommand{\bs}{\mathbf{s}}
\newcommand{\bt}{\mathbf{t}}
\newcommand{\bu}{\mathbf{u}}
\newcommand{\bv}{\mathbf{v}}
\newcommand{\bw}{\mathbf{w}}
\newcommand{\bx}{\mathbf{x}}
\newcommand{\by}{\mathbf{y}}
\newcommand{\bz}{\mathbf{z}}
\newcommand{\bzero}{\mathbf{0}}

% Proofs, Structures
\newcommand{\proof}{\tit{\underline{Proof:}}} % This equivalent to the \begin{proof}\end{proof} block
\newcommand{\proofforward}{\tit{\underline{Proof($\implies$):}}}
\newcommand{\proofback}{\tit{\underline{Proof($\impliedby$):}}}
\newcommand{\proofsuperset}{\tit{\underline{Proof($\supseteq$):}}}
\newcommand{\proofsubset}{\tit{\underline{Proof($\subseteq$):}}}
\newcommand{\contradiction}{$\longrightarrow\!\longleftarrow$}
\newcommand{\qed}{\hfill $\blacksquare$}

% Number Spaces, Vector Space
\newcommand{\R}{\mathbb{R}}
\newcommand{\real}{\mathbb{R}}
\newcommand{\complex}{\mathbb{C}}
\newcommand{\field}{\mathbb{F}}

% customized commands
\newcommand{\settag}[1]{\renewcommand{\theenumi}{#1}}
\newcommand{\tbf}[1]{\textbf{#1}}
\newcommand{\tit}[1]{\textit{#1}}
\newcommand{\largeover}[1]{\mkern 1.5mu\overline{\mkern-1.5mu#1\mkern-1.5mu}\mkern 1.5mu}
\newcommand{\double}[1]{\mathbb{#1}} % Set to behave like that on word
\newcommand{\trans}[3]{$#1:#2\rightarrow{}#3$}
\newcommand{\map}[3]{\text{$\left[#1\right]_{#2}^{#3}$}}
\newcommand{\dime}[1]{\mathrm{dim}(#1)}
\newcommand{\mat}[2]{M_{#1 \times #2}(\R)}
\newcommand{\aug}{\fboxsep=-\fboxrule\!\!\!\fbox{\strut}\!\!\!}
\newcommand{\basecase}{\textsc{\underline{Basis Case:}} }
\newcommand{\inductive}{\textsc{\underline{Inductive Step:}} }
\newcommand{\norm}[1]{\left\lVert#1\right\rVert}
\newcommand{\independent}{\perp \!\!\! \perp}

\newcommand{\mods}{\mathrm{Mods}}
\newcommand{\pr}{\mathrm{Pr}}
\newcommand{\ent}{\mathrm{ENT}}
\newcommand{\mi}{\mathrm{MI}}
\newcommand{\dsep}{\mathrm{dsep}}

\newtheorem{theorem}{Theorem}[chapter]
\newtheorem{corollary}{Corollary}[theorem]
\newtheorem{algorithm}{Algorithm}[chapter]
\newtheorem{proposition}{Proposition}[chapter]
\newtheorem{definition}{Definition}[chapter]
\newtheorem{lemma}[theorem]{Lemma}


% Set section number in front of equation enumerations
\counterwithin{equation}{section}
\counterwithin{footnote}{section}
\author{Tingfeng Xia}
\title{CS289a: Great Theory Hits of 21st Century}
\date{Winter 2023}

\begin{document}
\maketitle

\vspace*{\fill}
Notes reorganized from \url{https://hackmd.io/@raghum/greathits}.\newline \newline 
\doclicenseThis

\tableofcontents

\chapter{Undirected s-t Connectedness}

\section{Computing Resources}
Four main computing resources that we consider as limited (and measure the performance of our algorithms against)
\begin{itemize}
	\item Time 
	\item Memory
	\item Randomness
	\item Communication
\end{itemize}

\section{Problem Statement}

\begin{itemize}
	\item \textbf{Input}: Graph $G = (V, E)$; with source and target marked as $s, t$
	\item \textbf{Output}: YES iff $s$ and $t$ are connected, NO otw.
\end{itemize}

Above is the ``traditional'' definition of $s-t$ connectivity which we can solve with a vanilla BFS or DFS. This will take $\mathcal O ( |V| + |E| ) $ and $\mathcal O (|V| )$ extra bits of space / memory. The question is then, can we solve the same problem with sub-linear extra memory usage. 

\begin{proposition}
	There is a randomized algorithm with $5 \log |V|$ bits of additional memory (directed and undirected graphs). 
\end{proposition}

\begin{proposition}[Omer Reingold, 2005]
	There is a log space ($\mathcal O ( \log | V | )$) algorithm \textbf{(deterministic)} for undirected graphs. \footnote{first great hit ...}
\end{proposition}

It is yet unknown if we can achieve log space for directed graphs (with deterministic algorithm). The best known algorithms runs with $\mathcal O (\log |V| ) ^{3/2}$ bits of memory. Why is this so challenging? 

\begin{proposition}
	If divided $s-t$ connectivity can be solved with $\mathcal O (\log |V| )$ extra bits of memory (without randomness), then any randomized algorithm can be made deterministic at the expenses of a constant factor increase in memory. 
\end{proposition}

\section{Randomized Algorithm for Connectivity}
\begin{algorithm}[Random Walk Algorithm for Connectivity] Here is the algorithm
	\begin{itemize}
		\item $steps \gets 0 $ 
		\item $current \gets s$; $target \gets t$
		\item while $steps < T$
		\begin{itemize}
			\item $current \gets $ random neighbor of current
			\item if $current == target$ return $YES$
		\end{itemize}
		\item return $NO$
	\end{itemize}
\end{algorithm}
The total memory for this algorithm is 
\begin{equation}
	2 \log N + \log T \leq 5 \log N
\end{equation}
extra bits, assuming we can get random neighbor. 

\begin{proposition}[Alenilaus, 80s]
	If $T = 100N^3$ steps, then $Pr$[Algorithm wrong] < $\frac{1}{3}$
\end{proposition}
which can improved to arbitrary accuracy by repeating the algorithm. Algorithms of this nature can perform bad on graphs known as ``Lollipop Graphs'' and even worse a ``Dumbell Graph''


\section{Log Space USTCON}
Here we highlight the progression in space complexity in various papers
\begin{itemize}
	\item \textbf{Nisan, 92}: Space $\mathcal O (\log ^ 2 N)$, time $N ^{\mathcal O (1)}$ algorithm... improved to $\mathcal O (\log ^ {4/3} N)$ in space. 
	\item \textbf{Reingold, 05}: Space $\mathcal O (\log N)$, time $N ^{\mathcal O (1)}$ algorithm. 
	\item \textbf{Trifornov, 05}: Space $\mathcal O ((\log N )(\log\log N))$ algorithm. 
\end{itemize}


\section{Spectral Graph Theory}
Consider an undirected graph $G = (V, E)$, 
\begin{definition}[Degree]
	Degree of a vertex $v$ is the number f edges $v$ is connected to.
\end{definition}

\begin{definition}[Regular]
	Graphs is ``regular'' if all vertices have same degree. 
\end{definition}

\begin{definition}[Adjacency Matrix]
	$A(G)$ is a symmetric matrix where $A(G)_{ij}$ = 1 if $\{i, j\}$ is an edge, 0 otw. 
\end{definition}

\begin{definition}[Normalized Adj Matrix]
	If $G$ is regular and has degree $D$, then the normalized adjacency matrix is defined as 
	\begin{equation}
		M(G) \equiv \frac{A(G)}{D}
	\end{equation}
\end{definition}

\begin{lemma}
	If $G$ is regular, then $1$ is an eigenvalue of $M(G)$. And $\bv_1 = \begin{bmatrix}
		1 & 1 & \dots & 1
	\end{bmatrix}^\top$ is an eigenvector with eigenvalue 1. 
\end{lemma}

\begin{proposition}[Eigenvalues of Regular Graphs]
	If $G$ is regular, then all eigenvalues of $M(G)$ have magnitude $\leq 1$. 
\end{proposition}

\begin{proof}
	WLOG assume $x_3$ is the largest entry in the vector $\bx$, then
	\begin{align}
		\lambda |x_3| 
		&= |M_{31}x_1 + M_{32}x_2 + ... + M_{3N}x_N| \\
		&\leq M_{31}|x_3| + M_{32}|x_3| + ... + M_{3N} |x_3| \\
		&= (M_{31} + ... M_{3N}) |x_3| \\
		&= 1 |x_3|
	\end{align}
	Thus, $\lambda \leq 1$. \qed
\end{proof}

\begin{proposition}[Connectedness and Matrices]\todo{Might appear on exam 1}
	Regular $G = (V, E)$ is connected if and only if the only eigenvector with eigenvalue 1 for $M(G)$ is the all 1 vector. 
\end{proposition}

\begin{proposition}[Eigenvalues of a Regular Graph]\todo{Might appear on exam 1}
	If $G$ is regular, then the eigenvalues of $M(G)$ are 
	\begin{equation}
		1 = \lambda_1 \geq \lambda_2 \geq \dots \geq \lambda_N
	\end{equation}
\end{proposition}

\begin{proposition}\todo{Might appear on exam 1}
	$G$ is connected and regular if and only if 
	\begin{equation}
		\max ( |\lambda_2| , |\lambda_3|, \dots, |\lambda_N|) \leq 1
	\end{equation}
\end{proposition}

\begin{proposition}[Eigenvalues of D-Regular Graphs]
	If $G$ is a D-regular graph, then
	\begin{itemize}
		\item 1 is an eigenvalue of $M(G)$, and 
		\item all eigenvalues of $M(G)$ are at most 1 in absolute value
	\end{itemize}
\end{proposition}

\begin{definition}[Self Loops]
	We add connections from each node in the graph to themselves. In the matrix representation, we set $G_{ii} = 1, \forall i$. 
\end{definition}

\begin{definition}[Second Largest Eigenvalue]
	... denoted as $\lambda(G)$ or $\lambda_2(G)$. 
\end{definition}

\begin{lemma}
	If $G$ is D-regular and \textbf{has self loops}, then $G$ is connected if and only if $\lambda(G) < 1$.
\end{lemma}

\begin{proof} We first show $G$ is disconnected implies $\lambda(G) = 1$ (via contrapositive). 
	\todo{add later}
\end{proof}

\begin{definition}[Spectral Gap]
	Spectral Gap of a D-regular graph G is defined as 
	\begin{equation}
		\text{Spectral Graph} 
		\equiv 1 - \lambda(G)
	\end{equation}
\end{definition}

\begin{lemma}
	If $G$ is a D-regular connected graph with self-loops, then
	\begin{equation}
		\lambda (G) \leq 1 - \frac{1}{2D^2 \cdot N^2}
	\end{equation}
\end{lemma}

\begin{definition}
	We say a graph $G$ is $(N, D, \lambda)$ if it has $N$ vertices, $D$ regular and $\lambda(G) \leq \lambda$. 
\end{definition}


\section{Easy Cases}


\end{document}