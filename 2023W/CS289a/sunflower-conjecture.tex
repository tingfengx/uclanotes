\chapter{Sunflower Conjecture}

\section{Sunflower Lemma}

\begin{definition}
	[Sunflowers (Erd\"os-Rado, 1960's)]
	Take $f = \{ s_1 , \dots, s_m \}$ is a sunflower if all pairwise intersections are the same. i.e., 
	\begin{equation}
		S_i \cap S_j = S_k \cap S_e \quad \quad \text{if} \quad i \neq j, k \neq e.
	\end{equation}
	
	\textbf{\textit{Equivalently, }}
	$S_1, \dots, S_m$ is a sunflower if the ``core''
	\begin{equation}
		C = S_1 \cap S_2 \cap \dots \cap S_m
	\end{equation}
	are such that 
	\begin{equation}
		S_1 \text{\textbackslash} C, S_2 \text{\textbackslash} C, \dots, S_m \text{\textbackslash} C
	\end{equation}
	are disjoint
\end{definition}

\begin{lemma}
	[Sunflower Lemma (Erd\"os-Rado, 1962)] 
	Take $f = \{ S_1, \dots, S_m \} \subseteq [n]$, \footnote{
	This is a bit of abuse of notation. We really meant $S_i \subseteq [n], \forall i$ and $f$ is a set of such subsets. 
	}
	where $|S_i| \leq k, \forall i \in [m]$. Then $S$ contains a $r$-sunflower if 
	\begin{equation}
		m \geq (r - 1)^k \cdot k!
	\end{equation}
	
	\paragraph{English Explanation} This essentially says that any large collection of sets (large enough) must contain a sunflower. Interestingly enough, this expression does not depend on the size of universe, $n$. 
\end{lemma}

\begin{proof}
	The proof is an induction on $k$. 
	\paragraph{Base Case} $k = 1$. Then $S_i, S_j$ are disjoint and $m > (r - 1) \implies m \geq r$. So all of $f$ is a sunflower with empty core.
	\paragraph{Inductive Step} There are two cases to consider. Suppose the family $f$ had $ \geq r$ disjoint sets, then we are done. Otherwise, we choose some maximal collection of disjoint sets
	\begin{equation}
		S_{i1}, S_{i2}, ..., S_{il} \quad \quad \text{where} \quad l \leq r - 1
	\end{equation}
	This means that we cannot add another set that is disjoint from these. In particular, every other set must intersect with one of these sets. Now, call 
	\begin{equation}
		A \triangleq S_{i1} \cup S_{i2} \cup \dots \cup S_{il}
	\end{equation} 
	then the total number of elements in $A$ is bounded by 
	\begin{align}
		|A| 
		&\leq k \cdot l  \\
		&\leq k \cdot (r - 1) 
	\end{align}
	Now, every set $S_i$ must intersect $A$. Then, some element $a \in A$ must occur in $ \geq \alpha = m / |A|$ many sets. 
	\begin{align}
		\alpha 
		&\geq \frac{m}{k\cdot(r - 1)} \\
		&\geq \frac{(r - 1)^k \cdot k!}{k \cdot (r - 1)} \\
		&= (r - 1)^{k - 1} \cdot (k - 1)! \\
		\implies &m \geq (r - 1)^k \cdot k!
	\end{align}
	This concludes the proof. \qed
\end{proof}

\begin{proposition}
	[Erd\"os-Rado Conjecture]
	For every $r$, there is some constant $c_r$, such that if $m > c_r^k$, then it contains a $r$-sunflower. 
\end{proposition}

\begin{theorem}
	[ALWZ, 2020]
	There exists a constant $C$ such that if $f = \{ S_1, \dots, S_m \} \subseteq [n]$ and $|S_i| \leq k$, and if $m > (c \cdot r \cdot \log k ) ^k$, then $f$ contains a $r$-sunflower. 
\end{theorem}

\begin{definition}
	[Link of a Set System]
	Take $f = \{ S_1, \dots, S_m \} \subseteq [n]$. Suppose $I \subseteq [n]$, then define the link of $I$ in $f$ as 
	\begin{equation}
		f_i = \{ S_i : S_i \supseteq I \} \quad \quad \overline {f_I} = \{ S_i \setminus I : S_i \supseteq I \}
	\end{equation}
\end{definition}

\begin{definition}
	[Spread]
	A set system $f = \{ S_1, S_2, \dots, S_m \} \subseteq [n]$, where each $|S_i| \leq k$, is $s$-spread if 
	\begin{itemize}
		\item $|f| \geq s^k$, and 
		\item $\forall I \subseteq [n], |f_I| \leq s^{k - |I|} \cdot |f|$
	\end{itemize}
\end{definition}

\begin{lemma}
	[Main Lemma, ALWZ20]
	If $f$ is $s$-spread for 
	\begin{equation}
		s = c \cdot \log ( r k ) \log \log (r k)
	\end{equation}
	then $f$ contains $r$ disjoint sets. 
\end{lemma}

\begin{proposition}
	Main Lemma implies new sunflower lemma bounds. 	
\end{proposition}

\begin{proof}
	Suppose we have 
	\begin{equation}
		|f| \geq C(r, k)^k = s^k = (c \cdot \log ( r k ) \log \log (r k)) ^k
	\end{equation}
	If there exists a $I$ such that $|f_I| > s^{-|I|}$, then we are done. We see
	\begin{align}
		|f_I| 
		&\geq s^{-|I|} (c \cdot \log ( r k ) \log \log (r k))^k \\
		&= s^{k - |I|} \\
		&\geq \left( 
			c \cdot \log (r (k - |I|)) \cdot \log \log (r \cdot (k - |I|))
		\right)^{k - |I|}
	\end{align}
	so we rely on induction for this proof. If $\forall I$, $|f_I| < s^{- |I|} \cdot |f|$. Then, $f$ is $s$-spread ad by main lemma $f$ contains $r$ disjoint sets. \qed
\end{proof}

\section{Robust Sunflower Lemma}
\begin{lemma}
	[Robust Sunflower Lemma (RSL)]
	Consider $f$ as $s-spread$, and $w$ is a random subset of $[n]$ of size $\lfloor \frac{n}{r} \rfloor$. Then, 
	\begin{equation}
		Pr_{w} [\exists j, S_j \subseteq W] \geq 1 - \frac{1}{2r}
	\end{equation}
\end{lemma}

It is natural to wonder why we need this different form of the Sunflower Conjecture. In fact, this more robust RSL implies the Main lemma. 

The idea to the proof of RSL lies in the fact that we have flexibility in choosing $W$ (of size $n/r$, assuming divisibility), and we don't have to choose it in one shot. We choose $W$ as a sequence of sets
\begin{equation}
	W = V_1 \cup V_2 \cup \dots \cup V_t
\end{equation}
where each $V_i$ is a random subset of what's left of size $\frac{n}{rt}$. 

\paragraph{TODO} add lecture 9 from min 57. 
%\begin{definition}
%	[
%\end{definition}
%
%\begin{proposition}
%	Suppose that there exists $j$ such that $|\chi(j, V)| = 0$ then there exists some set that has been fully covered. 
%\end{proposition}
%
%\begin{proposition }
%	[Refined RSL]
%\end{proposition }


\begin{proposition}
	RSL $\implies$ MSL. 
\end{proposition}
This stronger form of the conjecture comes in handy when we are doing induction, we can have a more powerful hypothesis in the induction step. 

\begin{proof}
	Intuitively, consider the system to be a ``spread'' one, as then we can sample random patches $W$'s of the total space $[n]$, each of size $\lfloor \frac{n}{r} \rfloor$ elements. We expect that a patch covers some set $S_i$ with a reasonable chance ($(1 - \frac{1}{2r})$ stated in theorem). If this is the case, then if we take $r$ patches, we immediately know
	\begin{align}
		\text{RSL} \implies 
		\begin{cases}
			Pr[\exists j, S_j \subseteq W_1] = Pr[E_1] \geq 1 - \frac{1}{2r} \\
			Pr[\exists j, S_j \subseteq W_2] = Pr[E_2] \geq 1 - \frac{1}{2r} \\
			\quad \quad \vdots \\
			Pr[\exists j, S_j \subseteq W_r] = Pr[E_r] \geq 1 - \frac{1}{2r} 
		\end{cases}
	\end{align}
	Then, we can take the union bound of events to bound $Pr$[everything above happens at the same time] as 
	\begin{align}
		Pr[E_1 \cup E_2 \cup \dots \cup E_r] 
		&\leq Pr[E_1] + Pr_[E_2] + \dots + Pr[E_r] \\
		&= r \cdot \frac{1}{2r} \\
		&= \frac{1}{2} \\
		&> 0
	\end{align}
	So there $\exists$ $r$ disjoint sets. \qed
\end{proof}




















































